\documentclass{article}

\usepackage[letterpaper,top=2cm,bottom=2cm,left=2cm,right=2cm,marginparwidth=1.75cm]{geometry}

\usepackage{amsmath}
\usepackage{graphicx}
\usepackage{titlesec}
\usepackage[parfill]{parskip}
\usepackage[colorlinks=true, allcolors=blue]{hyperref}

\pdfpagewidth=8.5in
\pdfpageheight=11in


\def\tricolfig#1{\includegraphics[width=2.2in]{#1}}
\def\triplecolfig#1{\includegraphics[width=0.32\textwidth]{#1}}
\def\smallcolfig#1{\includegraphics[width=2.0in]{#1}}
\def\smallercolfig#1{\includegraphics[width=0.75\columnwidth]{#1}}
\def\mediumcolfig#1{\includegraphics[width=0.9\columnwidth]{#1}}
\def\colfig#1{\includegraphics[width=\columnwidth]{#1}}
\def\pagefig#1{\includegraphics[width=0.75\textwidth]{#1}}
\def\suspcolfig#1{\includegraphics[width=2.0in]{#1}}
\def\perfcolfig#1{\smallercolfig{#1}}



\def\crcd{compiler-runtime co-design}


\title{Compiler and Runtime Support for Functions-as-a-Service (FaaS)}
\date{}
\author{Souradip Ghosh, 15-849, Fall 2021}

\begin{document}
\maketitle

\section{Abstract}
Functions-as-a-Service (FaaS) are rapidly growing in popularity and 
complexity among cloud service provides as users look to take 
advantage of cloud resources on-demand and focus on computation
at a finer grain. FaaS instances are characteristically short-lived
and often dominated by communication and cold-start overheads compared
to typical monolithic applications. At the same time, FaaS instances
are requiring more resources, especially memory, to operate. Although
many works have approached these issues with scheduling optimizations,
hardware accelerators, and other techniques, very few have examined
\textit{compiler} support and \textit{\crcd} for FaaS. This paper presents 
a \crcd\ consisting of static analyses, profiling, and a custom memory 
allocator controlled by the compiler and runtime to better understand 
and optimize \textit{dynamic memory usage} for small FaaS workloads.

\section{Introduction/Background}

\section{Design}
The design of the system consists of three major components : 1) a \textit{profiler},
2) \textit{static analyses} in the compiler, and 3) a \textit{custom memory allocator}.
The first and third components consist of compiler instrumentation coupled with an 
integrated runtime. These compoents are described in the following subsections
in detail. 

\begin{figure}
    \centerline{\pagefig{figs/sys.pdf}}
    \caption{System overview and compilation pipeline. }  
	\label{fig:sys}
\end{figure}

An overview of the system and compilation pipeline is show in Figure~\ref{fig:sys}. The 
first pass over an example workload (\texttt{app.c}) consists of static analyses and 
profiling. The second pass over \texttt{app.c} instruments the program with the custom
allocator using the results from the static analyses (and optionally the profiler). The
outputs from this compilation pipeline include an instrumented application (\texttt{app.exe})
statistics about dynamic memory working set size.

\subsection{Profiler}

\subsection{Static Analyses}

\subsection{Custom Allocator}

\section{Implementation}

\section{Evaluation}

\section{Future Work}

\section{Conclusion}

\bibliographystyle{plain}
\bibliography{sample}

\end{document}
